\documentclass[11pt]{article}
\usepackage[sfdefault,semibold,lf]{FiraSans}
\usepackage[a4paper,margin=1in]{geometry}
\usepackage{xcolor}
\definecolor{accent}{HTML}{141E61}
\newcommand{\linkicon}{
  \color{gray}{\footnotesize\raisebox{1pt}{\faIcon{external-link-alt}}}
}
\setcounter{secnumdepth}{0}
\pdfgentounicode=1
%====== LIST FORMATTING ======
\usepackage{enumitem}
\setlist[itemize]{nosep, left=0pt .. 1.5em}
\setlist[enumerate]{left=0pt..1.5em}
\setlist[description]{itemsep=0pt}
%====== TITLE FORMATTING ======
\usepackage{xhfill}
\usepackage{soul}
\newcommand{\midrule}[1]{
  \leavevmode
  \xrfill[.5ex]{1pt}[accent]~\so{#1}~\xrfill[.5ex]{1pt}[accent]
}
\usepackage{titlesec}
\titlespacing{\section}{0pt}{*4}{*1}
\titlespacing{\subsection}{0pt}{*3}{*0}
\titlespacing{\subsubsection}{0pt}{*0}{*0}
\titleformat{\section}{\large\bfseries\uppercase}{}{}{\midrule}
\titleformat*{\subsection}{\large\bfseries\scshape}
%====== CUSTOM COMMANDS ======
\newcommand{\aux}[1]{
  $|$ \space {\normalfont\itshape #1}%
}
\newcommand{\rside}[1]{
  \hfill {\normalfont\color{accent} #1}%
}
%====== FOOTER ======
\usepackage{fontawesome5}
\usepackage{multicol}
\usepackage[bookmarks=false,hidelinks]{hyperref}
\usepackage{lastpage}
\usepackage{fancyhdr}
\usepackage{xcolor}
\usepackage{fontawesome5}
\usepackage{tabularx}
\usepackage{graphicx}
\pagestyle{fancy}
\fancyhf{}
\renewcommand{\headrulewidth}{0pt}
% Replace 'Star Rover' with your name in next line
\fancyfoot[C]{\small\color{gray} Ali Rıza Kat -- Page \thepage\ of \pageref*{LastPage}}
%====== DOCUMENT STARTS HERE ======
\begin{document}
%====== BANNER ======
\begin{center}
  \begin{tabularx}{\textwidth}{Xr}
    % Left-aligned name and links
    \begin{minipage}[c]{0.7\textwidth} % Center vertically
      {\fontsize{36}{12} \fontseries{heavy}\selectfont \color{accent} ALI RIZA KAT} \\[0.5em]
      \href{tel:+905516617781}{{\color{gray}{\faPhone}} +905516617781} \quad
      \href{mailto:alirizakat.dev@gmail.com}{{\color{gray}{\faEnvelope}} alirizakat.dev@gmail.com} \quad
      \faMapMarker \ {\color{gray} Istanbul} \\
      \href{https://github.com/AliRKat}{{\color{gray}{\faGithub}} AliRKat} \quad
      \href{https://www.linkedin.com/in/alirizakat}{{\color{gray}{\faLinkedin}} alirizakat} \quad
      \href{https://raw.githubusercontent.com/AliRKat/CV/main/Ali_Riza_Kat_Resume.pdf}{{\color{gray}{\faFilePdf }} İngilizce CV} \quad
    \end{minipage} &
    % Right-aligned photo
    \begin{minipage}[c]{0.25\textwidth} % Center vertically
      \centering
      \vspace{-5mm} % Adjust vertical alignment of the image
      \includegraphics[width=3cm, height=3cm, keepaspectratio]{photo.jpg} % Your photo path
    \end{minipage}
  \end{tabularx}
\end{center}
%====== BANNER END ======
\section*{Özet}
\begin{flushleft}
Yazılım ve Oyun Geliştiricisi olarak Unity, SDK geliştirme ve optimizasyon konusunda güçlü bir geçmişe sahibim. Oyun motorları, düşük seviyeli sistemler ve performans ayarları konusunda tutkuluyum. C++ ve Unreal Engine konusunda deneyimliyim ve PS2 ile PSP gibi eski donanımlar için geliştirme yapmaya büyük ilgi duyuyorum. Açık kaynak teknolojilerine hevesliyim, Arch Linux kullanıcısıyım ve her zaman yazılım ve donanımın sınırlarını zorlamanın yollarını araştırıyorum.
\end{flushleft}
%==========
\section{Deneyim}

\subsection{Multiplayer / Super Duper Games \rside{İstanbul, Türkiye}}
\subsubsection{Oyun Geliştirici \rside{Şubat 2025 -- Şu Anda}}
\begin{itemize}
  \item Rollic Games'in önde gelen stüdyolarından biri için prototipler ve projeler geliştiriyorum.
  \item Parserlar ve editör araçları gibi dahili araçlar geliştiriyor, ayrıca tabloları ve oyunla ilgili dokümanları düzenleyip yönetiyorum.
  \item Marketing videoları ve tanıtımlar için özel sahneler ve oyun seviyeleri tasarlıyorum.
  \item Prototipleri düzenli olarak değerlendirerek hataları tespit ediyor ve düzeltmelere destek oluyorum.
\end{itemize}

\subsection{Countly \rside{Uzaktan}}
\subsubsection{SDK Engineer \rside{Eylül 2023 -- Kasım 2024}}
\begin{itemize}
  \item Unity SDK'si için tüm zamanlarda en fazla kod geliştiren ve en geniş kapsamlı iyileştirmeleri gerçekleştiren kişi olarak görev aldım. WebGL desteği gibi önemli katkılarda bulundum.
  \item \textbf{Unity, React Native, and Node.JS} platformları için SDK'lerin geliştirilmesinden ve güncel versiyonların piyasaya çıkartılmasından sorumlu kişiydim.
  \item Müşteriler ve şirket içerisindeki diğer yazılımcılar için okunaklı ve anlaşılır dokümanlar hazırladım ve Countly'nin blogu için SDK'ler ile ilgili makaleler yazdım.
  \item SDK'lerin her fonksiyonalitesi için Mocha ve NUnit ile test yazımında bulundum.
\end{itemize}

\subsection{Nitrogen Games \rside{Izmir, Türkiye}}
\subsubsection{Oyun Geliştirici \rside{Kasım 2022 -- Mayıs 2023}}
\begin{itemize}
  \item Birçok oyun prototipi ve tamamlanan projenin tek yazılımcısı olarak geliştirilmesinde görev aldım.
  \item Şirket içi kullanılan oyun geliştirme araçlarının oluşturulmasına katkı sağladım.
  \item Geliştirilen projelerin ve prototiplerin test ediminde ve hatalarının ayıklanmasında rol aldım.
\end{itemize}

\subsection{Nirah Games  \rside{Izmir, Türkiye}}
\subsubsection{Oyun Geliştirici \rside{Ocak 2021 -- Ekim 2022}}
\begin{itemize}
  \item Oyun prototipleri ve tamamlanan projelerin geliştirilmesinde yazılımcı olarak görev aldım, ve tüm yazılım süreçlerini üstlendim.
  \item "Baby Life 3D!" adında \textbf{40 binden fazla inceleme ve milyonlarca indirme} almış bir projenin gelişiminde kilit bir rol üstlendim.
  \item 2 seneye yakın bir süreçte, geliştirilen çoğu projenin baştan sona yazılımcısı olarak bulundum.
\end{itemize}

\subsection{Freelance \rside{Remote}}
\subsubsection{Oyun Geliştirici \rside{Ocak 2020 -- Güncel}}
\begin{itemize}
  \item Farklı firmaların projelerine ve prototiplerine geliştirme desteğinde bulundum.
  \item Projelerin mekaniklerinin implement edilmesi, hatalarının ayıklanması ve çözülmesi, kullanıcı arayüzünün istenildiği gibi oluşturulmasında katkı sağladım
\end{itemize}
%===============
\section{Beceriler}
\begin{description}
   \item[Teknik] C\#, Node.js, React Native, JavaScript, C++, NUnit, WebGL, Mocha, SQL, Java, Kotlin
  \item[Araçlar] Unity, Android Studio, XCode, Git, Unity Version Control, Zendesk
  \item[Diller] İngilizce (Advanced)
\end{description}
%=======================
%=======================
\section{Projeler}

\subsubsection{Coin Commanders \aux{\href{https://apps.apple.com/us/app/coin-commanders/id6741897031?uo=2}{Link \linkicon}}}
\begin{itemize}
  \item Plinko mekaniğini geliştirdim ve sahnede on binlerce obje varken bile performans sorunlarını önlemek için optimize ettim.
  \item Oyunun Google Sheets üzerinden yapılan düzenlemeleri aktarabilmesi için parserlar ve alt sistemler yazdım.
\end{itemize}

\subsubsection{Baby Life 3D \aux{\href{https://apps.apple.com/us/app/baby-life-3d/id1577193350}{Link \linkicon}}}
\begin{itemize}
    \item Oyunun akışını optimize ederek, sezgisel bir kullanıcı arayüzü ve sorunsuz bir kullanıcı deneyimi sağladım.
    \item Yüksek etkileşim ve oyuncu tutma oranını sağlamak amacıyla güncel mobil oyun trendlerini entegre ettim.
\end{itemize}

\subsubsection{Subway Hero \aux{\href{https://apps.apple.com/us/app/subway-hero/id6741066164}{Link \linkicon}}}
\begin{itemize} \item Oyun testleri sürecine aktif olarak katılarak, özellikle gameplay ve UI tarafındaki hataları tespit edip detaylı şekilde raporladım.
\item Geliştirici ekiple birlikte çalışarak bug fix süreçlerinde destek verdim; yeniden üretilebilir senaryolar sağlayarak düzeltmelerin doğruluğunu test ettim. \end{itemize}

\subsubsection{Smith Journey \aux{\href{https://apps.apple.com/us/app/smith-journey/id6741895744}{Link \linkicon}}}
\begin{itemize} \item Oynanış testleri ve balans kontrolü sırasında uç senaryolardaki hataları belirleyerek geliştirme sürecine katkıda bulundum.
\item Sprint teslimatları öncesinde yapılan regresyon testlerinde yer alarak, build'lerin sorunsuz şekilde ilerlemesini sağladım. \end{itemize}

\subsubsection{Merge Invaders \aux{\href{https://apkcombo.com/merge-invaders/com.WorldnMore.MergeInvaders/}{Link \linkicon}}}
\begin{itemize}
    \item Oyunun temel döngüsüne yönelik merge mekaniğini geliştirdim ve uyguladım.
    \item Büyük sahne nesnelerini yönetmek için oyun performansını optimize ederek gecikmeyi azalttım ve kare hızlarını iyileştirdim ve farklı cihazlar için belleği ve sahne yönetimini optimize etmeye odaklandım.
\end{itemize}

\subsubsection{Island Explorer 3D \aux{\href{https://www.taptap.io/app/33559961}{Link \linkicon}}}
\begin{itemize}
    \item Yükseltmeler, oyun içi satın alımlar, kayıt sistemi ve dövüş mekanikleri gibi temel oyun mekaniklerini uyguladım.
    \item Etkileşimli ortamlar ve kaynak yönetimi özelliklerini entegre etmeye odaklandım.
\end{itemize}

\subsubsection{AniMayhem \aux{\href{https://www.appbrain.com/appstore/animayhem/ios-1579526762}{Link \linkicon}}}
\begin{itemize}
    \item Gerçekçi futbol ağı dinamikleri için özel bir kumaş simülasyonu kullanarak oyun içi fiziği optimize ettim.
    \item Çarpışma algılama ve fizik etkileşimlerini daha iyi performans ve görsel doğruluk için ayarladım.
\end{itemize}

\subsubsection{Party Queen \aux{\href{https://skich.app/games/party-queen}{Link \linkicon}}}
\begin{itemize}
    \item Oyuncu etkileşimini ve görsel çekiciliği artıran animasyonlar entegre ettim.
    \item Oyunun çeşitli cihazlarda sorunsuz çalışmasını ve geniş bir kitleye erişilebilir olmasını sağlamak için optimizasyon yaptım.
\end{itemize}

\subsubsection{Guess The Place \aux{\href{https://skich.app/games/guess-the-place}{Link \linkicon}}}
\begin{itemize}
    \item Kültürel ve coğrafi bilgiye dayalı bir trivia mekaniği tasarlayıp uyguladım.
    \item Oyuncu etkileşimini artırmak için sezgisel bir kullanıcı arayüzü ve kullanıcı deneyimi geliştirdim.
\end{itemize}

\subsubsection{Crazy In Love \aux{\href{https://skich.app/games/crazy-in-love}{Link \linkicon}}}
\begin{itemize}
    \item Yeniden oynanabilirliği artırmak için dinamik, hikaye tabanlı oynanış ve dallanan hikaye hatları geliştirdim.
    \item Oyuncu beklentilerine ve trendlere uygun oyun mekaniklerini detaylı bir şekilde iyileştirmeye odaklandım.
\end{itemize}

\subsubsection{City.io \aux{\href{https://skich.app/games/cityio}{Link \linkicon}}}
\begin{itemize}
    \item Gerçek zamanlı olarak oyuncuyla rekabet edebilen mantıklı ve aktif bir rakip oynanışı sağlayan durum tabanlı bir düşman yapay zeka sistemi geliştirdim.
    \item Kullanıcı memnuniyetini artıran ve dinamik görsel geri bildirim sağlayan görsel olarak çekici para toplama ve istifleme sistemi ile kaynak yönetimi mekaniklerini uyguladım.
\end{itemize}
%=======================
\section{Yayinlar}
\subsubsection{Countly SDK Development Updates: What's New and Improved \aux{\href{https://countly.com/blog/countly-sdk-development-updates-what-s-new-and-improved}{Link \linkicon}}}
\begin{itemize}
    \item Countly'nin SDK geliştirmeleri hakkında yapılan son güncellemeleri ve iyileştirmeleri kapsayan, geliştiricilerin yeni özellikleri anlamalarına ve bunları nasıl kullanabileceklerine yardımcı olmayı amaçlayan detaylı bir yazı yazdım. \\
\end{itemize}
\subsubsection{Unleashing the Unique Power of Countly's Desktop SDKs \aux{\href{https://countly.com/blog/unleashing-the-unique-power-of-countlys-desktop-sdks}{Link \linkicon}}}
\begin{itemize}
    \item Geliştiricilerin masaüstü uygulamalarına analiz entegrasyonu yaparken kullanabilecekleri, Countly'nin masaüstü SDK'larının benzersiz özelliklerini ve avantajlarını inceleyen bir makale yazdım. \\
\end{itemize}
%=======================
\section{Egitim}
\subsection{Yaşar Üniversitesi \rside{Izmir, Türkiye}}
\begin{itemize}
    \item{İşletme \rside{Mezuniyet: 2021}}
\end{itemize}
%=======================
\section{Ilgi Alanlarim}
\begin{itemize}
    \item \textbf{Dövüş Sporları:} Brazilian Jiu Jitsu, Taekwondo Siyah Kemer, Wing Chun 1.TG Eğitmen, 
    \item \textbf{Oyun:} Hayatım boyunca Satranç ve Magic: the Gathering oynadım. Aynı zamanda retro oyunlara ve konsollara büyük bir tutkum var, bu platformlar için hobi olarak oyun geliştiriyorum. 
    \item \textbf{Dungeons \& Dragons:} Hikaye yazımı ve anlatımı konusunda büyük bir ilgim var.
    \item \textbf{Koşu:} Mümkün olduğu sürece senede en azından bir maraton koşusuna katılmaya çalışıyorum.
    \item \textbf{Retro Oyunlar ve Geliştirme:} Retro konsolları ve platformları toplamak ve eski donanımlarda (örneğin, PS2, PSP) oyun geliştirmeyi keşfetmek. Donanım sınırlamaları içinde çalışmanın getirdiği teknik ve eğitici zorluklardan keyif alıyorum.
\end{itemize}
%=======================
\end{document}